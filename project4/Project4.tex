\documentclass[11pt, oneside]{article}   	% use "amsart" instead of "article" for AMSLaTeX format
\usepackage{geometry}                		% See geometry.pdf to learn the layout options. There are lots.
\geometry{letterpaper}                   		% ... or a4paper or a5paper or ... 
%\geometry{landscape}                		% Activate for for rotated page geometry
%\usepackage[parfill]{parskip}    		% Activate to begin paragraphs with an empty line rather than an indent
\usepackage{graphicx}				% Use pdf, png, jpg, or eps� with pdflatex; use eps in DVI mode
								% TeX will automatically convert eps --> pdf in pdflatex		
\usepackage{amssymb}
\usepackage{listings}

\title{CS 325: Project 4}
\author{Robert Erick, Jacob Mastel, Cera Olson}
\date{7 June 2015}							% Activate to display a given date or no date

\begin{document}
\maketitle
\newpage
\tableofcontents
\newpage
\section{The Algorithm}
\indent The algorithm used by our group to solve this problem was Prim's Algorithm. Prim's Algorithm is a greedy algorithm that finds the shortest distance between a list of nodes on an undirected, weighted graph. The is a common and effective algorithm used to solve the Traveling Salesman Problem (TSP). See Appendix 1 for the actual code we used to find our results. \\
\indent Our Algorithm is programmed in Python. We started by defining 2 global functions: aCity and getCities. aCity takes list of strings and converts them to integers, the cost of movement to the city. getCities takes the values returned in aCity and creates a list of these items. getCities is used in the different classes to find the neighboring cities and to determine the cost to their connections.\\
\indent We used three separate classes to separate out the different sections of the problem - the node, the minimum spanning tree, and the traveling salesman problem itself.  

\subsection{Node Class}
\indent The node class represents a single city. This class contains multiple functions that define the individual values of each node in a tree - distance, cost, neighbors, and location in the spanning tree. This class is what actually creates each node used in the minimum spanning tree. Node initates by first creating an id number, an x coordinate and a y coordinate and then adding them to a list. These values are later used in the MST Class.

\begin{itemize}
\item distance: Uses the equation $d = \sqrt(x^2 + y^2)$ to find the distance between this node and its neighbors. We chose to have the function return a round numbers, to the closest integer. This eases the calculations and any nodes that have a difference in distance less than 1, that difference is considered arbitrary in the overall problem. 
\item minEdge: Using the previously described distance function, the minEdge function creates a chart of edge weights, determining the minEdge between the current node and its neighbors. If the nodes not connected to the "self" node are within the set maximum distance (MAX\_CACHE\_DIST), it is added to the tables and lists the shortest distances between the cataloged nodes. 
\item preorder: Takes in a node and outputs a list of children that is then used in the TSP class. Because preorder uses extend(), each time a node is found or processed, they are added to the list.
\item printTree: A pretty straight forward function that prints out the resulting tree. 
\end{itemize}

\subsection{MST Class}
\indent MST is the class that is run in order to create the Minimum Spanning Tree. The output of this class is the input for the next class, TSP. MST is initiated with a node and a list of lists. Each sublist contains an id number, an x coordinate, and a y coordinate. These are the values put through the distance function in Node. The node class is called in the getMST function. 

\begin{itemize}
\item getMST: This function is what creates the minimum spanning tree off which the TSP solutions are based. It begins by sorting the nodes and then popping off the first value in the list. From there is systematically goes through the list of nodes, finds the minimum distances, and adds them to the list. By using a while loop based on nodes that aren't yet connected, it prevents nodes from being processed more than once.
\item extractMin: This is directly from Prim's Algorithm. This function is called in getMST and finds the shortest path between the nodes. Starting with an initial input node, to compares all the connected nodes, removes the unconnected nodes, and adds them to a list unique to the parent node. 
\end{itemize}

\subsection{TSP Class}	
\indent The TSP class takes the final output from the MST, the minimum spanning tree created from a list of nodes, and determine a route to connect all the nodes and the shortest route of the available options. The class is initiated by taking the spanning tree created by the MST class. Using the origin node and the last of connections, it initiates the MST class and creates a preorder array, a total variable and a presentation array. These are used in the functions within the class to determine the best route for the problem.

\begin{itemize}
\item getPreorder: Using the Node preorder function, to get all the neighboring nodes, adding them to the array.
\item getTotal: Starting at the initial input node, getTotal goes through the list and finds the shortest route to travel to all the nodes. This will later be compared to all the totals to find the shortest path. This function is operational but is written in a manner that it returns a working total but does not find other options. This means that our results may not be the best option available. 
\item getPresentation: This function takes the totals and adds them to the preorder array, listing all the data in one place.
\item fileLines: This function strictly prints out all the results in a file. 
\end{itemize}

\section{The Tours}
\indent The results are listed below comparing our results with the optimal solutions. There appears to be no correlation between the size of the spanning trees and the accuracy of our algorithm. Our Algorithm is correct and returns a legitimate route. However, there are areas that could be improved to find the optimal route. One area is discussed in TSP Class -> get Total. Given that the assignment was to find a route, and not to find the optimal route, our results will get the salmon to ever home - it will just take him a long time. 

\subsection{Tour 1}
\indent Our algorithm above came up with a total cost of 136,087. The optimal tour length is 108,159. Our algorithm takes 25\% longer than the optimal length. 

\subsection{Tour 2}
\indent The optimal cost of tour 2 is 2,579. Our results came up with a tour of 3,633 meaning our algorithm's route is 40\% slower than the optimal length. 

\subsection{Tour 3}
\indent Tour 3 has an optimal route of 1,573,084. Our algorithm route has a length of 4,105,325. Our route is 161\% longer than the optimal route. 

\subsection{Competition Tours}

\newpage
\section{Appendix}

\subsection{Appendix 1: Algorithm Code}
\lstinputlisting[breaklines = true]{main.py}
\subsection{Appendix 2: Example Results}
\subsubsection{Example 1}
\input{ex1Results.txt}
\subsubsection{Example 2}
\input{ex2Results.txt}
\subsubsection{Example 3}
\input{ex3Results.txt}







\end{document}  