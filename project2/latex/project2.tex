\documentclass[11pt, oneside]{amsart}   	% use "amsart" instead of "article" for AMSLaTeX format
\usepackage{geometry}                		% See geometry.pdf to learn the layout options. There are lots.
\geometry{letterpaper}                   		% ... or a4paper or a5paper or ... 
%\geometry{landscape}                		% Activate for for rotated page geometry
%\usepackage[parfill]{parskip}    		% Activate to begin paragraphs with an empty line rather than an indent
\usepackage{graphicx}				% Use pdf, png, jpg, or eps� with pdflatex; use eps in DVI mode
								% TeX will automatically convert eps --> pdf in pdflatex		
\usepackage{amssymb}
\usepackage{listings}

\title{CS 325: Project 2}
\author{Group 29: Jacob Mastel, Robert Erick, Cera Olson}
\date{10 May 2015}							% Activate to display a given date or no date

\begin{document}
\maketitle
\newpage{}


\section{Changeslow Algorithm}

\subsection{Pseudocode}

\lstinputlisting[language=Python]{pseudocode/pseudocode_a1.txt}



\section{Greedy Algorithm}

\subsection{Pseudocode}

\lstinputlisting[language=Python]{pseudocode/pseudocode_a2.txt}



\section{Dynamic Programming}

\subsection{Pseudocode}

\lstinputlisting[language=Python]{pseudocode/pseudocode_a3.txt}

\section{Questions}
\subsection{Describe, in words, how you fill in the dynamic programming table in changedp. Justify why is this a valid way to fill the table?}


\subsection{Give pseudocode for each algorithm.}
See Sections 1-3. 


\subsection{Prove that the dynamic programming approach is correct by induction. That is, prove that
$T[v] = min_{v(i)�v}{T[v-V[i]] + 1}, T[0] = 0$ is the minimum number of coins possible to make change for value v.}


\subsection{Suppose V = [1, 5, 10, 25, 50]. For each integer value of A in [2010, 2015, 2020, ..., 2200] determine the number of coins that changegreedy and changedp requires. You can attempt to run changeslow however if it takes too long you can select smaller values of A and also run the other algorithms on the values. Plot the number of coins as a function of A for each algorithm. How do the approaches compare?}


\subsection{Suppose $V_1$ = [1, 2, 6, 12, 24, 48, 60] and V2 = [1, 6, 13, 37, 150]. For each integer value of A in [2000, 2001, 2002, ..., 2200] determine the number of coins that changegreedy and changedp requires. If your algorithms run too fast try [10,000, 10,001, 10,003, ..., 10,100]. You can attempt to run changeslow however if it takes too long you can select smaller values of A and also run all three algorithms on the values. Plot the number of coins as a function of A for each algorithm. How do the approaches compare?}


\subsection{Suppose V = [1, 2, 4, 6, 8, 10, 12, ..., 30]. For each integer value of A in [2000, 2001, 2002, ..., 2200] determine the number of coins that changegreedy and changedp requires. You can attempt to run changeslow however if it takes too long you can select smaller values of A and also run all three algorithms on the values. Plot the number of coins as a function of A for each algorithm.


\subsection{For the above situations, determine (experimentally) the running times of the algorithms by fitting trend lines to the data or analyzing the log-log plot. Graph the running time as a function of A. Compare the running times of the different algorithms.}


\subsection{Use the data from questions 4-6 and any new data you have generated. Plot running times as a function of number of denominations (i.e. V=[1, 10, 25, 50] has four different denominations so n=4). Does the size of n influence the running times of any of the algorithms?}


\subsection{Suppose you are living in a country where coins have values that are powers of p, $V = [p^0 , p^1 , p^2 , ��� , p^n]$. How do you think the dynamic programming and greedy approaches would compare? Explain}


\newpage{}
\section{Appendices}
\subsection{Code}


\subsection{Tests}
		

\end{document}  